En considérant la matrice de Hilbert $A \in \R^{n \times n}$

\begin{equation*}
  A = \begin{bmatrix}
        1                & 1/2      & 1/3      & 1/4      & \dots \\
        1/2      & 1/3      & 1/4      & \dots            &       \\
        1/3      & \vdots           &                  &                  &       \\
        \vdots           &                  &                  &                  &      
      \end{bmatrix}
      \quad \quad 
      A_{ij} = \dfrac{1}{i+j-1}
      ;
\end{equation*}

\begin{enumerate}[label=\alph*)]
  \item Calculer avec \textsc{MATLAB} les éléments de la matrice $A$ avec $n = 10$ (en utilisant la commande \texttt{hilb}) et le nombre de conditionnement $K_{2}\parent{A}$ avec la commande \texttt{cond}.
        Enregistrer les nombres de conditionnement calculés pour $n = 1, \dots , 10$ dans un vecteur avec 10 composantes.
        
  \item Visualiser sur un graphe en échelle semilogarithmique (linéaire sur les abscisses et logarithmique sur les ordonnées) la valeur de $K_{2}\parent{A}$ en fonction de $n$, pour $1 \leq n \leq 10$ en utilisant la commande \texttt{semilogy}.
        En déduire que $K_{2}\parent{A}$ se comporte comme $e^{\alpha n}$.
        
  \item Construire un vecteur colonne $\BoldX_{e x}$ aléatoire avec $n$ composantes en utilisant la commande \texttt{rand} et calculer le vecteur $\BoldB = A \BoldX_{e x}$.
        Pour chaque $n = 1, 2, \dots , 10$ résoudre le système linéaire $A \BoldX = \BoldB$ avec la commande \textbackslash \ qui met en oeuvre une méthode directe, et calculer l'erreur relative
  
        \begin{equation}
        \label{eq:err}
          \varepsilon^{r} = \dfrac{\norm{\BoldX - \BoldX_{e x}}}{\norm{\BoldX_{e x}}}.
        \end{equation}
        
        Visualiser l'erreur sur un graphe en échelle semilogarithmique et en déduire que cette erreur se conduit comme $K_{2}\parent{A}$.
        Visualiser ensemble l'erreur relative $\varepsilon^{r}$ et son estimation $\tilde{\varepsilon}^{r}$ en échelle semilogarithmique, où $\BoldR$ est le résidu et $\tilde{\varepsilon}^{r}$ est défini comme
        
        \begin{equation}
        \label{eq:tildeerr}
          \tilde{\varepsilon}^{r}
          = K_{2}\parent{A} \frac{\norm{\BoldR}}{\norm{\BoldB}}.
        \end{equation}
        
         
        
\end{enumerate}


