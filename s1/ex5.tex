\begin{enumerate}[label=\alph*)]
  \item Soit
  
  \begin{equation*}
    f \parent{x} = \dfrac{x^{2}}{2} \sin \parent{x}, 
    \quad x \in \squared{1, 20}
  \end{equation*}
  
  une fonction qu'on veut représenter graphiquement en choisissant 10 points, 20 points et 100 points dans l'intervalle de définition. Écrire un fichier ".m" pour réaliser les trois graphiques sur la même figure et avec trois couleurs différentes. Quelle est la meilleure représentation ?
  
  \item Faire la même chose pour les fonctions :
  
  \begin{equation*}
    g \parent{x} = \dfrac{x^{3}}{6} \cos \parent{\sin \parent{x}} \exp\bracket{-x} + \parent{\dfrac{1}{1+x}}^{2}, 
    \quad x \in \squared{1, 20}
  \end{equation*}
  
  \begin{equation*}
    h \parent{x} = x \parent{1 - x} + \dfrac{\sin \parent{x} \cos \parent{x}}{x^3},
    \quad x \in \squared{1, 20}
  \end{equation*}
\end{enumerate}



